% ===================================================================
% CHAPTER 3: The EV2Gym Simulation Framework
% ===================================================================
\chapter{An Enhanced V2G Simulation Framework for Robust Control}
\label{chap:ev2gym}

Developing, validating, and benchmarking advanced control algorithms for Vehicle-to-Grid (V2G) systems is a task fraught with complexity. Real-world experimentation is often impractical due to prohibitive costs, logistical challenges, and risks to grid stability and vehicle hardware. To bridge the gap between theory and practice, a realistic, flexible, and standardized simulation environment is a scientific necessity. This thesis builds upon the foundation of \textbf{EV2Gym}, a state-of-the-art, open-source simulator designed for V2G smart charging research \footcite{orfanoudakis2024ev2gym}. However, this work extends the original framework significantly, transforming it into a high-fidelity \textbf{digital twin} engineered not just for single-scenario optimization, but for the development and rigorous evaluation of \textbf{robust, generalist control agents}.

This enhanced framework provides a dual-pronged approach to experimentation: it allows for deep-dive analysis of agents specialized for a single environment, while also introducing a novel methodology for training and testing agents designed to generalize across a multitude of diverse, unpredictable scenarios. This chapter provides an in-depth tour of this extended architecture, its data-driven models, and its unique evaluation capabilities, establishing the methodological bedrock for the rest of this work.

\section{Core Simulator Architecture}
The framework retains the modular architecture of EV2Gym, which mirrors the key entities of a real-world V2G system. Its foundation on the OpenAI Gym (now Gymnasium) API remains a cornerstone, providing a standardized agent-environment interface defined by the familiar language of states, actions, and rewards \footcite{brockman2016openai}.

The architecture consists of several interacting components:
\begin{itemize}
    \item \textbf{Charge Point Operator (CPO):} The central intelligence of the simulation, managing the charging infrastructure and serving as the primary interface for the control algorithm (the DRL agent). The CPO aggregates system state information and dispatches control actions to individual chargers.
    \item \textbf{Chargers:} Digital representations of physical charging stations, configurable by type (AC/DC), maximum power, and efficiency. This allows for the simulation of heterogeneous charging infrastructures.
    \item \textbf{Power Transformers:} These components model the physical connection points to the grid, aggregating the electrical load from multiple chargers. Crucially, they enforce the physical power limits of the local distribution network and can model inflexible base loads (e.g., buildings) and local renewable generation (e.g., solar panels).
    \item \textbf{Electric Vehicles (EVs):} Dynamic and autonomous agents, each defined by its unique battery capacity, power limits, current and desired energy levels, and specific arrival and departure times.
\end{itemize}

The simulation process follows a reproducible three-phase structure: (1) \textbf{Initialization} from a comprehensive YAML configuration file, (2) a discrete-time \textbf{Simulation Loop} where the agent interacts with the environment, and (3) a final \textbf{Evaluation and Visualization} phase that generates standardized performance metrics.

\section{Core Physical Models}
The fidelity of the simulation is anchored in its detailed and empirically validated models, which are essential for developing control strategies robust enough for real-world application.

\subsection{EV Model and Charging/Discharging Dynamics}
The framework implements a realistic two-stage charging/discharging model that captures the non-linear behavior of lithium-ion batteries, simulating both the \textbf{constant current (CC)} and \textbf{constant voltage (CV)} phases. Each EV is defined by a rich parameter set: maximum capacity ($E_{max}$), a minimum safety capacity ($E_{min}$), separate power limits for charging and discharging ($P_{ch}^{max}, P_{dis}^{max}$), and distinct efficiencies for each process ($\eta_{ch}, \eta_{dis}$).

\subsection{Battery Degradation Model}
To address the critical issue of battery health in V2G operations, the simulator incorporates a semi-empirical battery degradation model. It quantifies capacity loss ($Q_{lost}$) as the sum of two primary aging mechanisms \footcite{orfanoudakis2024ev2gym}:
\begin{itemize}
    \item \textbf{Calendar Aging ($d_{cal}$):} Time-dependent capacity loss, influenced by the battery's average State of Charge (SoC) and temperature.
    \item \textbf{Cyclic Aging ($d_{cyc}$):} Wear resulting from charge/discharge cycles, dependent on energy throughput, depth-of-cycle, and C-rate.
\end{itemize}
This integrated model allows for the direct quantification of how different control strategies impact the battery's long-term State of Health (SoH), enabling the training of agents that balance profitability with battery preservation.

\subsection{EV Behavior and Grid Models}
To ensure realism, the simulation is driven by authentic, open-source datasets. EV arrival/departure patterns and energy requirements are modeled using probability distributions derived from a large real-world dataset from \textbf{ElaadNL}. Grid conditions are similarly grounded in reality, using inflexible load data from the \textbf{Pecan Street} project and solar generation profiles from the \textbf{Renewables.ninja} platform \footcite{orfanoudakis2024ev2gym}.

\section{A Dual-Pronged Evaluation Architecture}
A key contribution of this thesis is the development of a sophisticated, dual-mode evaluation pipeline, which distinguishes between specialized and generalized agent performance. This is implemented through two primary execution scripts: \texttt{Single\_Domain\_Env.py} and \texttt{MultiScenarioEnv.py}.

\subsection{Single-Domain Specialization}
The \texttt{Single\_Domain\_Env.py} script is designed to train and evaluate "specialist" agents. In this workflow, a Reinforcement Learning agent is trained from scratch on a single, fixed configuration file. This approach is used to answer the question: "What is the optimal performance achievable for this specific, known environment?" It allows for a deep-dive analysis of an agent's ability to master one particular scenario, serving as a crucial baseline for performance.

\subsection{Multi-Scenario Generalization}
The \texttt{MultiScenarioEnv.py} script introduces a more challenging and realistic paradigm: training a single, "generalist" agent that must perform well across a diverse set of scenarios. This is achieved through two key innovations:
\begin{itemize}
    \item \textbf{MultiScenarioEnv:} A custom Gymnasium environment that acts as a wrapper around multiple underlying \texttt{EV2Gym} instances. At the beginning of each training episode (i.e., on \texttt{reset()}), this environment randomly selects one of the provided configuration files. This forces the agent to learn a robust policy that is not overfitted to any single scenario's characteristics (e.g., number of chargers, grid capacity, or price volatility).
    \item \textbf{CompatibilityWrapper:} A critical technical solution to handle the varying observation and action space sizes across different scenarios. Since a neural network policy has a fixed input and output size, this wrapper \textbf{pads} observations from smaller environments to a maximum size and \textbf{slices} action vectors from the agent to match the specific needs of the currently active environment. This enables a single agent to seamlessly control infrastructures of varying scales.
\end{itemize}
This multi-scenario training methodology is fundamental to developing agents that are truly robust and ready for deployment in the real world, where conditions are never static.

\section{Software and Experimentation Workflow}
The project's functionality is organized into a modular structure to facilitate clear and reproducible experimentation.
\begin{itemize}
    \item \texttt{ev2gym/}: The core directory containing the simulator's heart.
    \begin{itemize}
        \item \texttt{models/}: Defines the main environment (\texttt{ev2gym\_env.py}) and the physical components (\texttt{ev.py}, \texttt{ev\_charger.py}, \texttt{transformer.py}).
        \item \texttt{baselines/}: Contains the classical control algorithms used for benchmarking, including heuristics (\texttt{heuristics.py}) and Model Predictive Control (\texttt{pulp\_mpc.py}).
        \item \texttt{rl\_agent/}: Houses DRL-specific components, such as state space definitions (\texttt{state.py}) and reward functions (\texttt{reward.py}).
        \item \texttt{data/}: Contains the input time-series data for EV arrivals, energy prices, and loads.
    \end{itemize}
    \item \texttt{Compare.py}: A powerful utility script for pre-analysis and scenario comparison. It reads multiple YAML configuration files and generates summary tables and legends as images, allowing for a quick, visual comparison of experimental setups.
    \item \texttt{Single\_Domain\_Env.py}: The primary script for training and evaluating specialist agents on a single, user-selected scenario. It orchestrates the entire benchmark for one environment.
    \item \texttt{MultiScenarioEnv.py}: The script for training and evaluating robust, generalist agents. It utilizes the \texttt{MultiScenarioEnv} to train a single agent on a collection of scenarios and then evaluates its performance across each of them.
\end{itemize}

\section{Evaluation Metrics}
To ensure a fair and comprehensive comparison, all algorithms are evaluated against the same set of pre-generated scenarios (using a "replay" mechanism). The \textbf{mean} and \textbf{standard deviation} of performance are calculated across multiple simulation runs. The key metrics include:

\begin{itemize}
    \item \textbf{Total Profit (\$):} The net economic outcome, calculated as revenue from energy sales minus the cost of energy purchases.
    \[
    \Pi_{\text{total}} = \sum_{t=0}^{T_{\text{sim}}} \sum_{i=1}^{N} \left( C_{\text{sell}}(t) P_{\text{dis},i}(t) - C_{\text{buy}}(t) P_{\text{ch},i}(t) \right) \Delta t
    \]
    
    \item \textbf{Tracking Error (RMSE, kW):} For grid-balancing scenarios, this measures the root-mean-square error between the fleet's aggregated power and a target setpoint.
    \[
    E_{\text{track}} = \sqrt{\frac{1}{T_{\text{sim}}} \sum_{t=0}^{T_{\text{sim}}-1} \left( P_{\text{setpoint}}(t) - P_{\text{total}}(t) \right)^2}
    \]
    
    \item \textbf{User Satisfaction (Average):} The fraction of energy delivered compared to what was requested by the user, averaged across all EV sessions. A score of 1 indicates perfect service.
    \[
    US_{\text{avg}} = \frac{1}{N_{\text{EVs}}} \sum_{k=1}^{N_{\text{EVs}}} \min \left(1, \frac{E_k(t_k^{\text{dep}})}{E_k^{\text{des}}} \right)
    \]
    
    \item \textbf{Transformer Overload (kWh):} The total energy that exceeded the transformer's rated power limit. An ideal controller should achieve a value of 0.
    \[
    O_{\text{tr}} = \sum_{t=0}^{T_{\text{sim}}} \sum_{j=1}^{N_T} \max(0, P_j^{\text{tr}}(t) - P_j^{\text{tr,max}}) \cdot \Delta t
    \]
    
    \item \textbf{Battery Degradation (\$):} The estimated monetary cost of battery aging due to both cyclic and calendar effects.
    \[
    D_{\text{batt}} = \sum_{k=1}^{N_{\text{EVs}}} (\text{CyclicCost}_k + \text{CalendarCost}_k)
    \]
\end{itemize}

\section{Reinforcement Learning Formulation}
The control problem is formalized as a Markov Decision Process (MDP), defined by the tuple $(S, A, P, R, \gamma)$.

\subsection{State Space ($S$)}
The state $s_t \in S$ is a feature vector providing a snapshot of the environment at time $t$. A representative state, as defined in modules like \texttt{V2G\_profit\_max\_loads.py}, includes:
\[
s_t = [t, P_{\text{total}}(t-1), \mathbf{c}(t, H), \mathbf{L}_1(t, H), \mathbf{PV}_1(t, H), \dots, \mathbf{s}^{\text{EV}}_1(t), \dots, \mathbf{s}^{\text{EV}}_N(t)]^T
\]
where the components are:
\begin{itemize}
    \item $t$: The current time step.
    \item $P_{\text{total}}(t-1)$: The aggregated power from the previous time step.
    \item $\mathbf{c}(t, H)$: A vector of \textbf{predicted future} electricity prices over a horizon $H$.
    \item $\mathbf{L}_j(t, H), \mathbf{PV}_j(t, H)$: Forecasts for inflexible loads and solar generation.
    \item $\mathbf{s}^{\text{EV}}_i(t) = [\text{SoC}_i(t), t^{\text{dep}}_i - t]$: Key information for each EV $i$, including its State of Charge and remaining time until departure.
\end{itemize}

\subsection{Action Space ($A$)}
The action $a_t \in A$ is a continuous vector in $\mathbb{R}^N$, where $N$ is the number of chargers. For each charger $i$, the command $a_i(t) \in [-1, 1]$ is a normalized value that is translated into a power command:
\begin{itemize}
    \item If $a_i(t) > 0$, the EV is charging: $P_i(t) = a_i(t) \cdot P^{\text{max}}_{\text{charge}, i}$.
    \item If $a_i(t) < 0$, the EV is discharging (V2G): $P_i(t) = a_i(t) \cdot P^{\text{max}}_{\text{discharge}, i}$.
\end{itemize}

\subsection{Reward Function ($R(s, a, s')$)}
The reward function $R(t)$ encodes the objectives of the control agent. The framework allows for the selection of different reward functions from the \texttt{reward.py} module to suit various goals. Key examples include:
\begin{itemize}
    \item \textbf{Profit Maximization with Penalties} (\texttt{ProfitMax\_TrPenalty\_UserIncentives}): This function creates a balance between economic gain and physical constraints.
    \[
    R(t) = \underbrace{\text{Profit}(t)}_{\text{Economic Gain}} - \underbrace{\lambda_1 \cdot \text{Overload}(t)}_{\text{Grid Penalty}} - \underbrace{\lambda_2 \cdot \text{Unsatisfaction}(t)}_{\text{User Penalty}}
    \]
    The agent is rewarded for profit but penalized for overloading transformers and for failing to meet the charging needs of departing drivers.
    
    \item \textbf{Squared Tracking Error} (\texttt{SquaredTrackingErrorReward}): Used for grid service applications where precision is paramount.
    \[
    R(t) = - \left( P_{\text{setpoint}}(t) - \sum_{i=1}^N P_i(t) \right)^2
    \]
    The reward is the negative squared error from the power setpoint, incentivizing the agent to minimize this error at all times.
\end{itemize}

By leveraging this enhanced framework, this thesis moves beyond single-scenario optimization to develop and validate an intelligent V2G control agent that is not only high-performing but also robust, adaptable, and ready for the complexities of real-world deployment.

\subsection{A History-Based Adaptive Reward for Profit Maximization}
\label{sec:adaptive_reward}

To effectively steer the learning agent towards a policy that is both highly profitable and reliable, we have designed and implemented a novel, history-based adaptive reward function, named \texttt{FastProfitAdaptiveReward}. This function departs from traditional static-weight penalties and instead introduces a dynamic feedback mechanism where the severity of penalties is directly influenced by the agent's recent performance. The core philosophy is to aggressively prioritize economic profit while using adaptive penalties as guardrails that become stricter only when the agent begins to consistently violate operational constraints.

The total reward at each timestep $t$, $R_t$, is calculated as the net economic profit minus any active penalties for user dissatisfaction or transformer overload.

\begin{equation}
    R_t = \Pi_t - P_t^{\text{sat}} - P_t^{\text{tr}}
\end{equation}

\subsubsection{Economic Profit}
The foundation of the reward signal is the direct, instantaneous economic profit, $\Pi_t$. This component provides a clear and strong incentive for the agent to learn market dynamics, encouraging it to charge during low-price periods and discharge (V2G) during high-price periods.
\begin{equation}
    \Pi_t = \sum_{i=1}^{N} \left( C_t^{\text{sell}} \cdot P_{i,t}^{\text{dis}} - C_t^{\text{buy}} \cdot P_{i,t}^{\text{ch}} \right) \Delta t
\end{equation}
where $N$ is the number of connected EVs, $C_t^{\text{sell}}$ and $C_t^{\text{buy}}$ are the electricity prices, and $P_{i,t}^{\text{dis}}$ and $P_{i,t}^{\text{ch}}$ are the discharging and charging powers for EV $i$.

\subsubsection{Adaptive User Satisfaction Penalty}
The penalty for failing to meet user charging demands, $P_t^{\text{sat}}$, is not a fixed value. Instead, it adapts based on the system's recent history of performance. The environment maintains a short-term memory of the average user satisfaction over the last 100 timesteps. From this history, we calculate an average satisfaction score, $\bar{S}_{hist}$.

A \textit{satisfaction severity multiplier}, $\lambda_t^{\text{sat}}$, is then calculated. This multiplier grows quadratically as the historical average satisfaction drops, meaning that if the system has been performing poorly, the consequences for a new failure become much more severe.
\begin{equation}
    \lambda_t^{\text{sat}} = \lambda_{\text{base}}^{\text{sat}} \cdot (1 - \bar{S}_{hist})^2
\end{equation}
where $\lambda_{\text{base}}^{\text{sat}}$ is a base scaling factor (e.g., 20.0). A penalty is only applied if any departing EV's satisfaction, $S_k$, is below a critical threshold (e.g., 95\%). The magnitude of the penalty is the product of the adaptive multiplier and the current satisfaction deficit.
\begin{equation}
    P_t^{\text{sat}} = \lambda_t^{\text{sat}} \cdot (1 - \min(S_k)) \quad \forall k \in \text{EVs departing at } t
\end{equation}
This creates a powerful feedback loop: a single, isolated failure in an otherwise well-performing system results in a mild penalty. However, persistent failures lead to a rapidly escalating penalty, forcing the agent to correct its behavior.

\subsubsection{Adaptive Transformer Overload Penalty}
Similarly, the transformer overload penalty, $P_t^{\text{tr}}$, adapts based on the recent frequency of overloads. The environment tracks how often an overload has occurred in the last 100 timesteps, yielding an overload frequency, $F_{hist}^{\text{tr}}$.

This frequency is used to compute a linear \textit{overload severity multiplier}, $\lambda_t^{\text{tr}}$. The more frequently overloads have happened, the higher the penalty for a new one.
\begin{equation}
    \lambda_t^{\text{tr}} = \lambda_{\text{base}}^{\text{tr}} \cdot F_{hist}^{\text{tr}}
\end{equation}
where $\lambda_{\text{base}}^{\text{tr}}$ is a base scaler (e.g., 50.0). If the total power drawn, $P_j^{\text{total}}(t)$, exceeds the transformer's limit, $P_j^{\text{max}}$, a penalty is applied. This penalty consists of a small, fixed base amount plus the adaptive component, which scales with the magnitude of the current overload.
\begin{equation}
    P_t^{\text{tr}} = P_{\text{base}} + \lambda_t^{\text{tr}} \cdot \sum_{j=1}^{N_T} \max(0, P_j^{\text{total}}(t) - P_j^{\text{max}})
\end{equation}
This mechanism teaches the agent that while a rare, minor overload might be acceptable in pursuit of high profit, habitual overloading is an unsustainable and heavily penalized strategy.

\subsubsection{Rationale and Significance}
This history-based adaptive reward function represents a significant advancement over static or purely state-based approaches. By making the penalty weights a function of the system's recent performance history, we provide a more nuanced and stable learning signal. The agent is not punished excessively for isolated, exploratory actions that might lead to a minor constraint violation. Instead, it is strongly discouraged from developing policies that lead to chronic system failures.

The intuition is to mimic a more realistic management objective: maintain high performance on average, and react strongly only when performance trends begin to degrade. This method is also computationally efficient, avoiding complex state-dependent calculations in favor of simple updates to historical data queues. Ultimately, this reward structure guides the agent to discover policies that are not only profitable but also robust and reliable over time, striking a more intelligent balance between economic ambition and operational safety.

\newpage
\section{Model Predictive Control (MPC)}
The MPC, implemented in \texttt{mpc.py} and \texttt{eMPC.py}, solves an optimization problem at every time step over a prediction horizon $H$.

\subsection{System Model}
The system is modeled in linear state-space form. The state $\mathbf{x}_k \in \mathbb{R}^N$ is the vector of SoCs of all EVs at time $k$. The input $\mathbf{u}_k \in \mathbb{R}^{2N}$ is the vector of charging and discharging powers.
\[
\mathbf{x}_{k+1} = A_k \mathbf{x}_k + B_k \mathbf{u}_k
\]
The matrices $A_k$ (\texttt{Amono}) and $B_k$ (\texttt{Bmono}) are time-varying because they depend on which EVs are connected. $A_k$ is typically a diagonal identity-like matrix modeling the persistence of EVs. $B_k$ maps power to SoC change, including efficiencies and $\Delta t$.

\subsection{Optimization Problem}
At time $t$, the MPC solves:
\[
\min_{\{\mathbf{u}_k\}_{k=t}^{t+H-1}} \sum_{k=t}^{t+H-1} \mathbf{f}_k^T \mathbf{u}_k
\]
subject to:
\begin{align*}
    & \mathbf{x}_{k+1} = A_k \mathbf{x}_k + B_k \mathbf{u}_k, \quad \forall k \in [t, t+H-1] & \text{(Dynamics)} \\
    & \mathbf{x}^{\text{min}}_k \le \mathbf{x}_k \le \mathbf{x}^{\text{max}}_k & \text{(SoC limits)} \\
    & \mathbf{0} \le \mathbf{u}^{\text{ch}}_k \le \mathbf{u}^{\text{ch,max}}_k \cdot \mathbf{z}_k & \text{(Charge limits)} \\
    & \mathbf{0} \le \mathbf{u}^{\text{dis}}_k \le \mathbf{u}^{\text{dis,max}}_k \cdot (1 - \mathbf{z}_k) & \text{(Discharge limits)} \\
    & \sum_{i \in \text{CS}_j} (u^{\text{ch}}_i - u^{\text{dis}}_i) + L_j(k) - PV_j(k) \le P_j^{\text{tr,max}}(k) & \text{(Transformer limits)}
\end{align*}
where $\mathbf{z}_k$ is a vector of binary variables to prevent simultaneous charge and discharge. The cost vector $\mathbf{f}_k$ contains the energy prices. The code formulates this problem compactly as $\mathbf{AU} \le \mathbf{bU}$, where $\mathbf{U}$ is the vector of all actions over the horizon.

\section{Offline Optimization with Gurobi}
Gurobi is used to find the optimal offline (a posteriori) solution, providing a performance benchmark. The files \texttt{profit\_max.py} and \texttt{tracking\_error.py} define the optimization problem over the entire simulation horizon $T_{\text{sim}}$.

\subsection{Decision Variables}
\begin{itemize}
    \item $E_{p,i,t}$: Energy in the EV at port $p$ of station $i$ at time $t$.
    \item $I^{\text{ch}}_{p,i,t}, I^{\text{dis}}_{p,i,t}$: Charging/discharging currents.
    \item $\omega^{\text{ch}}_{p,i,t}, \omega^{\text{dis}}_{p,i,t}$: Binary variables for operating modes.
\end{itemize}

\subsection{Objective Function (Example: Profit Maximization)}
\[
\max \sum_{t=0}^{T_{\text{sim}}} \sum_{i=1}^{N_{CS}} \sum_{p=1}^{N_p} \left( C_{\text{sell}}(t) P^{\text{dis}}_{p,i,t} - C_{\text{buy}}(t) P^{\text{ch}}_{p,i,t} \right) \Delta t - \lambda \sum_{k \in \text{EVs departed}} (E_k^{\text{des}} - E_k(t_k^{\text{dep}}))^2
\]
where $P = V \cdot I \cdot \eta$.

\subsection{Main Constraints}
\begin{itemize}
    \item \textbf{Energy Balance:}
    \[
    E_{p,i,t} = E_{p,i,t-1} + (\eta_{\text{ch}} V_i I^{\text{ch}}_{p,i,t} - \frac{1}{\eta_{\text{dis}}} V_i I^{\text{dis}}_{p,i,t}) \Delta t
    \]
    \item \textbf{Activation of Current:}
    \[
    I^{\text{ch}}_{p,i,t} \le M \cdot \omega^{\text{ch}}_{p,i,t} \quad , \quad I^{\text{dis}}_{p,i,t} \le M \cdot \omega^{\text{dis}}_{p,i,t}
    \]
    \item \textbf{Mutual Exclusion:}
    \[
    \omega^{\text{ch}}_{p,i,t} + \omega^{\text{dis}}_{p,i,t} \le 1
    \]
    \item \textbf{Current and SoC Limits:}
    \[
    I^{\text{min}} \le I_{p,i,t} \le I^{\text{max}} \quad , \quad E^{\text{min}} \le E_{p,i,t} \le E^{\text{max}}
    \]
    \item \textbf{SoC at Departure:}
    \[
    E_{p,i}(t^{\text{dep}}) \ge E^{\text{des}}_{p,i}
    \]
\end{itemize}
\section{Online MPC Formulation (PuLP Implementation)}

The Model Predictive Control (MPC) implemented with PuLP solves a profit maximization problem at each time step $t$ over a finite prediction horizon $H$. This formulation is designed for online, real-time control, where decisions are made based on the current system state and future predictions.

\subsection{Mathematical Formulation}
At each time step $t$, the MPC controller solves the following optimization problem.

\subsubsection{Objective Function: Net Operational Profit}
The objective is to maximize the total net operational profit over the control horizon $H$. This provides a comprehensive economic model that goes beyond simple energy arbitrage.
\begin{equation}
\max_{P^{\text{ch}}, P^{\text{dis}}, z} \sum_{k=t}^{t+H-1} \sum_{i \in \text{CS}} \left( \text{Revenues}_{i,k} - \text{Costs}_{i,k} \right)
\end{equation}
The revenue and cost components are defined for each station $i$ at time step $k$ as:
\begin{itemize}
    \item \textbf{Revenues} consist of:
    \begin{itemize}
        \item Grid Sales Revenue (V2G): $c^{\text{sell}}_k \cdot P^{\text{dis}}_{i,k} \cdot \Delta t$
        \item User Charging Revenue: $c^{\text{user}} \cdot P^{\text{ch}}_{i,k} \cdot \Delta t$
    \end{itemize}
    \item \textbf{Costs} consist of:
    \begin{itemize}
        \item Grid Purchase Cost: $c^{\text{buy}}_k \cdot P^{\text{ch}}_{i,k} \cdot \Delta t$
        \item Battery Degradation Cost: $c^{\text{deg}} \cdot (P^{\text{ch}}_{i,k} + P^{\text{dis}}_{i,k}) \cdot \Delta t$
    \end{itemize}
\end{itemize}
where $c^{\text{sell}}_k$ and $c^{\text{buy}}_k$ are the time-varying electricity prices, $c^{\text{user}}$ is the fixed price for the end-user, $c^{\text{deg}}$ is the estimated cost of battery degradation per kWh cycled, and $\Delta t$ is the time step duration.

\subsubsection{System Constraints}
The optimization is subject to the following constraints for each station $i$ and time step $k \in [t, t+H-1]$.

\paragraph{Energy Balance Dynamics.} The state of energy of the EV battery evolves according to:
\begin{equation}
E_{i,k} = E_{i,k-1} + \left( \eta^{\text{ch}} P^{\text{ch}}_{i,k} - \frac{1}{\eta^{\text{dis}}} P^{\text{dis}}_{i,k} \right) \cdot \Delta t
\end{equation}
where the initial state $E_{i,t-1}$ is the currently measured energy level of the EV.

\paragraph{Power Limits and Mutual Exclusion.} Charging and discharging powers are bounded by the EV's capabilities and controlled by a binary variable $z_{i,k}$ to prevent simultaneous operation.
\begin{align}
    0 &\le P^{\text{ch}}_{i,k} \le P^{\text{ch,max}}_{i} \cdot z_{i,k} \\
    0 &\le P^{\text{dis}}_{i,k} \le P^{\text{dis,max}}_{i} \cdot (1 - z_{i,k})
\end{align}

\paragraph{State of Energy (SoE) Limits.} The battery energy level must remain within its physical operational window.
\begin{equation}
E^{\text{min}}_{i} \le E_{i,k} \le E^{\text{max}}_{i}
\end{equation}

\paragraph{User Satisfaction (Hard Constraint).} The desired energy level must be met at the time of departure. This is modeled as a hard constraint, reflecting a non-negotiable service requirement.
\begin{equation}
E_{i,k_{\text{dep}}} \ge E^{\text{des}}_{i}
\end{equation}
where $k_{\text{dep}}$ is the predicted departure step of the EV within the horizon.

\paragraph{Transformer Power Limit.} The total net power drawn from (or injected into) the grid by all charging stations must not exceed the transformer's maximum capacity.
\begin{equation}
\sum_{i \in \text{CS}} (P^{\text{ch}}_{i,k} - P^{\text{dis}}_{i,k}) \le P^{\text{tr,max}}
\end{equation}


\section{Approximate Explicit MPC: A Machine Learning Approach}
The online, implicit MPC formulation provides high-quality control decisions by solving an optimization problem at every time step. However, this approach has a significant drawback: its computational complexity. For scenarios with a large number of EVs or a long control horizon, solving a Mixed-Integer Linear Program (MILP) in real-time can be prohibitively slow, making it impractical for many real-world applications.

To overcome this limitation, this work implements an \textbf{Approximate Explicit Model Predictive Controller (A-MPC)}. This controller leverages machine learning to replace the computationally expensive online optimization with a fast, lightweight inference step.

\subsection{Methodology: From Oracle to Apprentice}
The core idea is to treat the slow but powerful online MPC as an "oracle" or expert teacher. An apprentice model, in this case a \texttt{RandomForestRegressor} from the \texttt{scikit-learn} library, is trained to mimic the oracle's behavior. The process is as follows:
\begin{enumerate}
    \item \textbf{Data Generation:} The online MPC is run across a diverse range of simulated scenarios. At each step, the state of the environment and the corresponding optimal action computed by the MPC are recorded. This creates a large dataset of state-action pairs, where the actions are considered to be the "ground truth" optimal decisions.
    \item \textbf{State Vector Formulation:} The state $s_t$ fed to the machine learning model is a carefully crafted vector that summarizes all necessary information for making a control decision. It is a fixed-size vector composed of:
    \begin{equation}
        s_t = [ \mathbf{SoC}, \mathbf{T}^{\text{rem}}, \mathbf{C}^{\text{ch}}, \mathbf{C}^{\text{dis}} ]^T
    \end{equation}
    where:
    \begin{itemize}
        \item $\mathbf{SoC}$: A vector of the current State of Charge for all charging stations (padded to a maximum size).
        \item $\mathbf{T}^{\text{rem}}$: A vector of the remaining time until departure for each connected EV.
        \item $\mathbf{C}^{\text{ch}}$: A vector of predicted future charging prices over the horizon $H$.
        \item $\mathbf{C}^{\text{dis}}$: A vector of predicted future discharging prices over the horizon $H$.
    \end{itemize}
    \item \textbf{Offline Training:} The \texttt{RandomForestRegressor} model, denoted $f_{\theta}$, is trained offline on this dataset to learn the mapping from a given state $s_t$ to the oracle's action $a_t$. The model's parameters $\theta$ are optimized to minimize the difference between its predicted action and the oracle's action.
    \item \textbf{Online Inference:} Once trained, the A-MPC controller can be deployed. At each time step, it simply constructs the state vector $s_t$ and computes the action via a fast forward pass through the trained model:
    \begin{equation}
        a_t = f_{\theta}(s_t)
    \end{equation}
    This inference step is orders of magnitude faster than solving a MILP, enabling real-time control for large-scale systems.
\end{enumerate}

\section{Lyapunov-based Adaptive Horizon MPC}
While the A-MPC offers a significant speed-up, it is an approximation and may not always match the performance of the fully-fledged online MPC. A second enhancement developed in this work is the \textbf{Lyapunov-based Adaptive Horizon MPC}, which aims to reduce the computational burden of the online MPC while retaining its optimality and stability guarantees. This method represents an essential improvement, creating an intelligent trade-off between computational cost and control performance.

\subsection{Core Concept: Dynamic Horizon Adjustment}
The key insight is that a long prediction horizon is not always necessary. When the system is in a stable state and far from its operational constraints, a shorter horizon is sufficient for making good decisions. Conversely, when the system is in a complex or critical state (e.g., an EV is close to its departure time but has a low SoC), a longer horizon is needed for careful planning.

This adaptive controller dynamically adjusts its prediction horizon $H_t$ at each step based on the stability of the system, which is formally assessed using a Lyapunov function.

\subsection{Lyapunov Stability for V2G Control}
A Lyapunov function $V(x)$ is a scalar function that can be thought of as a measure of the system's "energy" or deviation from a desired equilibrium state. For the V2G system, we define the state as the vector of energy levels of all connected EVs, $E = [E_1, E_2, \dots, E_N]^T$. The desired state is the vector of desired energy levels at departure, $E^{\text{des}}$. The Lyapunov function is defined as the sum of the squared errors from this desired state:
\begin{equation}
    V(E) = \sum_{i \in \text{EVs}} (E_i - E_i^{\text{des}})^2
\end{equation}
For the system to be stable, the value of this function must decrease at each step, ensuring the system is always progressing towards its goal. This is known as the Lyapunov decrease condition:
\begin{equation}
    V(E_{t+1}) \le V(E_t) - \alpha V(E_t)
\end{equation}
where $E_{t+1}$ is the state at the next time step resulting from the current control action, and $\alpha$ is a small positive constant that sets the minimum required rate of convergence.

\subsection{Horizon Shortening and Extension}
The adaptive MPC algorithm uses this stability condition to govern its horizon length. At each time step $t$:
\begin{enumerate}
    \item The MPC solves the optimization problem using its current horizon, $H_t$.
    \item It calculates the predicted next state $E_{t+1}$ based on the computed optimal action.
    \item It checks if the Lyapunov decrease condition is satisfied.
    \begin{itemize}
        \item \textbf{If Stable:} The condition holds. The controller is performing well. We can afford to reduce the computational load for the next step by shortening the horizon:
        \begin{equation}
            H_{t+1} = \max(H_{\min}, H_t - 1)
        \end{equation}
        \item \textbf{If Not Stable:} The condition is violated. The system requires more careful planning. The horizon for the next step is extended to provide a longer view into the future:
        \begin{equation}
            H_{t+1} = \min(H_{\max}, H_t + 1)
        \end{equation}
    \end{itemize}
\end{enumerate}
This intelligent adjustment makes the online MPC more efficient and practical, reducing computation time during stable periods while retaining the ability to perform deep planning when necessary.