% ===================================================================
% CHAPTER 1: INTRODUCTION
% ===================================================================
\chapter{Introduction}
The shift toward electric mobility is a pivotal element in global strategies for decarbonizing transport. However, the large-scale integration of electric vehicles into existing power grids presents a complex set of challenges and opportunities, which this thesis aims to investigate.


\subsection{Background and Relevance of Electric Vehicles and Vehicle-to-Grid} % 
The surging Electric Vehicle (EV) market is accelerating a major reconfiguration of modern mobility, promising to lower carbon emissions and foster greater energy efficiency \footcite{orfanoudakis2022deep}. 
This evolution is more than a technological trend: it underpins environmental sustainability by reducing dependence on fossil resources, alleviating the impacts of climate change through diminished greenhouse gas emissions, and improving air quality in densely populated areas. 
However, embedding millions of EVs into existing power systems is a non-trivial task. 
It can intensify peak demand, place additional stress on transmission and distribution networks, and trigger side effects such as voltage irregularities or higher line losses \footcite{orfanoudakis2022deep, salvatti2020electric}.
\\
\noindent
It is in this context that the \textbf{Vehicle-to-Grid (V2G)} concept emerges as a strategic solution. 
Through bidirectional power exchange, V2G redefines EVs: no longer passive electrical loads, but mobile and flexible energy assets, able to deliver a spectrum of services to the power system \footcite{alfaverh2022optima}. 
This potential becomes even more compelling when one considers that, on average, EVs remain parked and unused for nearly 96\% of the day, offering an ample time window to actively engage with the grid \footcite{evertsson2024investigating}. 
A further distinctive benefit lies in the rapid responsiveness of EV batteries, which makes them especially suitable for ancillary services demanding quick interventions, such as frequency regulation \footcite{alfaverh2022optima}. 
Alongside V2G, other schemes of bidirectional power flow have been proposed, each with its own scope: 
\begin{enumerate}
    \item 
\textbf{Vehicle-to-Home (V2H)}, where an EV sustains household demand during outages or periods of elevated prices, strengthening domestic energy resilience;
\item
\textbf{Vehicle-to-Building (V2B)}, extending this logic to commercial or industrial facilities, enabling EVs to support load management and improve consumption efficiency; and 
\item
\textbf{Vehicle-to-Vehicle (V2V)}, which allows direct power transfer among EVs, a valuable feature for emergency charging or shared resources. 
\end{enumerate}
Taken together, these modalities highlight the versatility of EV batteries as distributed energy units, reinforcing both energy resilience and the transition toward a more sustainable energy ecosystem.

\subsection{Challenges in EV Integration into the Electricity Grid and the Role of Artificial Intelligence} % MODIFIED SECTION
Modern electricity systems are increasingly shaped by the penetration of intermittent \textbf{Renewable Energy Sources (RESs)} such as wind and solar. 
Their variability causes pronounced swings in power generation and persistent mismatches between supply and demand, which in turn fuels price volatility and complicates dispatch strategies. 
This continuously puts the stability and economic efficiency of the grid under strain. 
Managing these fluctuations, while making rapid operational choices to balance the system and minimize costs, has proven difficult for conventional control frameworks \footcite{orfanoudakis2022deep, minchala2025systematic}. 
\\
\noindent
The parallel rise of EV adoption and RES deployment has produced an environment marked by both uncertainty and complexity. 
In such conditions, traditional approaches are increasingly inadequate, prompting a growing reliance on methods rooted in artificial intelligence—and particularly in \textbf{Reinforcement Learning (RL)}. 
This shift alters the very nature of the grid: from a relatively predictable and centralized infrastructure to one that is decentralized, stochastic, and highly dynamic. 
Rule-based or deterministic controllers, designed for a past paradigm, are ill-suited to cope with this degree of volatility. 
The result is a pressing need for more adaptive and intelligent decision-making mechanisms. 
This transformation extends beyond the simple challenge of absorbing extra load or integrating new generators: it signals a genuine paradigm change towards a \emph{smart grid} \footcite{alhmoud2024review}, where adaptive, real-time, and autonomous operation is no longer optional but vital to preserve efficiency, resilience, and reliability. 
From this perspective, RL is not just an optimization tool, but an enabling technology for a more cognitive and robust energy infrastructure—one capable of navigating the uncertainties of a decarbonized, electrified future.
\\
\noindent
Against this backdrop, \textbf{Deep Reinforcement Learning (DRL)} has gained attention as an especially powerful approach. 
Its capacity to derive near-optimal strategies in dynamic and uncertain environments—without requiring a precise model of the system or flawless forecasts—makes DRL particularly well-suited for EV integration and advanced grid management \footcite{orfanoudakis2022deep, shibl2023electric}.


%====================================================================
% SEZIONE AGGIORNATA: OBIETTIVI E CONTRIBUTI
%====================================================================
\subsection{Objectives and Contributions of the Thesis}

This thesis confronts the complex multi-objective optimization problem at the heart of Vehicle-to-Grid (V2G) systems. The overarching objective is to move beyond a purely theoretical analysis by actively developing, testing, and enhancing a high-fidelity simulation architecture. This platform serves as a digital twin to rigorously evaluate and compare advanced control strategies, balancing economic benefits, user mobility needs, battery health, and grid stability under realistic stochastic conditions.
\\
More than a simple review of existing literature, this work focuses on the practical implementation and validation of a V2G simulation framework in Python. This tool is leveraged to demonstrate and explore novel perspectives for training intelligent agents. The main contributions are:

\begin{itemize}
    \item \textbf{Enhancement of a V2G Simulation Architecture:} A key contribution is the systematic testing, validation, and enhancement of the \textbf{EV2Gym} simulation framework. This work solidifies its role as a robust and flexible platform for benchmarking control algorithms, ensuring that the models for battery physics, user behavior, and grid dynamics are coherent and realistic for advanced research.

    \item \textbf{Exploration of Novel Reinforcement Learning Perspectives:} The validated simulation environment is used to investigate and implement advanced training methodologies for RL agents. A key focus is placed on techniques like \textbf{adaptive reward shaping}, where the reward function dynamically evolves during training to guide the agent towards a more holistic and robust control policy, overcoming the limitations of static reward definitions.

    \item \textbf{Practical Implementation of Advanced Control Paradigms:} The thesis demonstrates the practical transition from a theoretical, offline optimal controller to a realistic, online controller. Specifically, it details the implementation of an \textbf{offline MPC using Gurobi}, which acts as a "judge" with perfect foresight, and contrasts it with an \textbf{online MPC formulated in PuLP}, designed to operate as a real-time "controller" with limited future information, highlighting the trade-offs and challenges of real-world deployment.
\end{itemize}

\subsection{Research Methodology}
The research conducted for this thesis was guided by a central question: how can the economic profit of Vehicle-to-Grid (V2G) operations be maximized without imposing undue costs in terms of battery degradation or creating instability by overloading local grid infrastructure? To address this, a systematic research methodology was adopted.

The initial phase consisted of an extensive literature review, primarily leveraging Google Scholar as a search engine for academic papers. This review focused on identifying the state-of-the-art in V2G control strategies, with a particular emphasis on identifying the common trade-offs between profitability and physical constraints.

Following the literature review, this work transitioned to an empirical phase centered on the simulation framework detailed in Chapter 3. The development and testing of various control agents were not performed in a vacuum. The significance of the obtained results was continuously evaluated through a multi-faceted validation process, which included:
\begin{itemize}
    \item \textbf{Comparative Analysis:} Benchmarking the performance of novel agents against the results reported by other researchers in the field.
    \item \textbf{Heuristic Evaluation:} Assessing the outcomes based on the experience and intuition developed over the course of the study, ensuring that the results were not only quantitatively strong but also qualitatively sound.
    \item \textbf{Academic Grounding:} Constantly relating the empirical results back to the foundational knowledge acquired during the literature review to understand the underlying reasons for an agent's success or failure.
\end{itemize}
This iterative process of exploration, implementation, and critical evaluation forms the methodological backbone of the contributions presented in this thesis.


%====================================================================
% SEZIONE AGGIORNATA: STRUTTURA DELLA TESI
%====================================================================
\newpage
\subsection{Thesis Structure}
The remainder of this thesis is organized as follows:
\begin{itemize}
    \item \textbf{Chapter 2: Overview of Optimal Management of EV Charging and Discharging} provides foundational knowledge on V2G technology, the complex multi-objective nature of EV charging optimization, and presents a comprehensive review of state-of-the-art research approaches.

    \item \textbf{Chapter 3: The V2G Simulation Framework: A Digital Twin for V2G Research} details the architecture and core models of the simulation environment. This chapter describes the enhancements made to the framework, establishing it as the central experimental platform for implementing and evaluating the control agents analyzed in this work.

    \item \textbf{Chapter 4: Experimental Campaign and Results Analysis } This chapter presents the results of the comparative analysis between the different control strategies (DRL, MPC, heuristics). It analyzes the performance of novel training techniques and discusses the implications of the findings. % <-- Si consiglia di aggiungere un capitolo dedicato ai risultati

    \item \textbf{Bibliography} lists all cited references.
\end{itemize}